% Options for packages loaded elsewhere
\PassOptionsToPackage{unicode}{hyperref}
\PassOptionsToPackage{hyphens}{url}
%
\documentclass[
  man,floatsintext]{apa6}
\usepackage{amsmath,amssymb}
\usepackage{iftex}
\ifPDFTeX
  \usepackage[T1]{fontenc}
  \usepackage[utf8]{inputenc}
  \usepackage{textcomp} % provide euro and other symbols
\else % if luatex or xetex
  \usepackage{unicode-math} % this also loads fontspec
  \defaultfontfeatures{Scale=MatchLowercase}
  \defaultfontfeatures[\rmfamily]{Ligatures=TeX,Scale=1}
\fi
\usepackage{lmodern}
\ifPDFTeX\else
  % xetex/luatex font selection
\fi
% Use upquote if available, for straight quotes in verbatim environments
\IfFileExists{upquote.sty}{\usepackage{upquote}}{}
\IfFileExists{microtype.sty}{% use microtype if available
  \usepackage[]{microtype}
  \UseMicrotypeSet[protrusion]{basicmath} % disable protrusion for tt fonts
}{}
\makeatletter
\@ifundefined{KOMAClassName}{% if non-KOMA class
  \IfFileExists{parskip.sty}{%
    \usepackage{parskip}
  }{% else
    \setlength{\parindent}{0pt}
    \setlength{\parskip}{6pt plus 2pt minus 1pt}}
}{% if KOMA class
  \KOMAoptions{parskip=half}}
\makeatother
\usepackage{xcolor}
\usepackage{graphicx}
\makeatletter
\def\maxwidth{\ifdim\Gin@nat@width>\linewidth\linewidth\else\Gin@nat@width\fi}
\def\maxheight{\ifdim\Gin@nat@height>\textheight\textheight\else\Gin@nat@height\fi}
\makeatother
% Scale images if necessary, so that they will not overflow the page
% margins by default, and it is still possible to overwrite the defaults
% using explicit options in \includegraphics[width, height, ...]{}
\setkeys{Gin}{width=\maxwidth,height=\maxheight,keepaspectratio}
% Set default figure placement to htbp
\makeatletter
\def\fps@figure{htbp}
\makeatother
\setlength{\emergencystretch}{3em} % prevent overfull lines
\providecommand{\tightlist}{%
  \setlength{\itemsep}{0pt}\setlength{\parskip}{0pt}}
\setcounter{secnumdepth}{-\maxdimen} % remove section numbering
% Make \paragraph and \subparagraph free-standing
\ifx\paragraph\undefined\else
  \let\oldparagraph\paragraph
  \renewcommand{\paragraph}[1]{\oldparagraph{#1}\mbox{}}
\fi
\ifx\subparagraph\undefined\else
  \let\oldsubparagraph\subparagraph
  \renewcommand{\subparagraph}[1]{\oldsubparagraph{#1}\mbox{}}
\fi
\newlength{\cslhangindent}
\setlength{\cslhangindent}{1.5em}
\newlength{\csllabelwidth}
\setlength{\csllabelwidth}{3em}
\newlength{\cslentryspacingunit} % times entry-spacing
\setlength{\cslentryspacingunit}{\parskip}
\newenvironment{CSLReferences}[2] % #1 hanging-ident, #2 entry spacing
 {% don't indent paragraphs
  \setlength{\parindent}{0pt}
  % turn on hanging indent if param 1 is 1
  \ifodd #1
  \let\oldpar\par
  \def\par{\hangindent=\cslhangindent\oldpar}
  \fi
  % set entry spacing
  \setlength{\parskip}{#2\cslentryspacingunit}
 }%
 {}
\usepackage{calc}
\newcommand{\CSLBlock}[1]{#1\hfill\break}
\newcommand{\CSLLeftMargin}[1]{\parbox[t]{\csllabelwidth}{#1}}
\newcommand{\CSLRightInline}[1]{\parbox[t]{\linewidth - \csllabelwidth}{#1}\break}
\newcommand{\CSLIndent}[1]{\hspace{\cslhangindent}#1}
\ifLuaTeX
\usepackage[bidi=basic]{babel}
\else
\usepackage[bidi=default]{babel}
\fi
\babelprovide[main,import]{english}
% get rid of language-specific shorthands (see #6817):
\let\LanguageShortHands\languageshorthands
\def\languageshorthands#1{}
% Manuscript styling
\usepackage{upgreek}
\captionsetup{font=singlespacing,justification=justified}

% Table formatting
\usepackage{longtable}
\usepackage{lscape}
% \usepackage[counterclockwise]{rotating}   % Landscape page setup for large tables
\usepackage{multirow}		% Table styling
\usepackage{tabularx}		% Control Column width
\usepackage[flushleft]{threeparttable}	% Allows for three part tables with a specified notes section
\usepackage{threeparttablex}            % Lets threeparttable work with longtable

% Create new environments so endfloat can handle them
% \newenvironment{ltable}
%   {\begin{landscape}\centering\begin{threeparttable}}
%   {\end{threeparttable}\end{landscape}}
\newenvironment{lltable}{\begin{landscape}\centering\begin{ThreePartTable}}{\end{ThreePartTable}\end{landscape}}

% Enables adjusting longtable caption width to table width
% Solution found at http://golatex.de/longtable-mit-caption-so-breit-wie-die-tabelle-t15767.html
\makeatletter
\newcommand\LastLTentrywidth{1em}
\newlength\longtablewidth
\setlength{\longtablewidth}{1in}
\newcommand{\getlongtablewidth}{\begingroup \ifcsname LT@\roman{LT@tables}\endcsname \global\longtablewidth=0pt \renewcommand{\LT@entry}[2]{\global\advance\longtablewidth by ##2\relax\gdef\LastLTentrywidth{##2}}\@nameuse{LT@\roman{LT@tables}} \fi \endgroup}

% \setlength{\parindent}{0.5in}
% \setlength{\parskip}{0pt plus 0pt minus 0pt}

% Overwrite redefinition of paragraph and subparagraph by the default LaTeX template
% See https://github.com/crsh/papaja/issues/292
\makeatletter
\renewcommand{\paragraph}{\@startsection{paragraph}{4}{\parindent}%
  {0\baselineskip \@plus 0.2ex \@minus 0.2ex}%
  {-1em}%
  {\normalfont\normalsize\bfseries\itshape\typesectitle}}

\renewcommand{\subparagraph}[1]{\@startsection{subparagraph}{5}{1em}%
  {0\baselineskip \@plus 0.2ex \@minus 0.2ex}%
  {-\z@\relax}%
  {\normalfont\normalsize\itshape\hspace{\parindent}{#1}\textit{\addperi}}{\relax}}
\makeatother

% \usepackage{etoolbox}
\makeatletter
\patchcmd{\HyOrg@maketitle}
  {\section{\normalfont\normalsize\abstractname}}
  {\section*{\normalfont\normalsize\abstractname}}
  {}{\typeout{Failed to patch abstract.}}
\patchcmd{\HyOrg@maketitle}
  {\section{\protect\normalfont{\@title}}}
  {\section*{\protect\normalfont{\@title}}}
  {}{\typeout{Failed to patch title.}}
\makeatother

\usepackage{xpatch}
\makeatletter
\xapptocmd\appendix
  {\xapptocmd\section
    {\addcontentsline{toc}{section}{\appendixname\ifoneappendix\else~\theappendix\fi\\: #1}}
    {}{\InnerPatchFailed}%
  }
{}{\PatchFailed}
\keywords{IAS-R; Interoception; Personality; Psychopathology; Validation\newline\indent Word count: 1249}
\usepackage{lineno}

\linenumbers
\usepackage{csquotes}
\usepackage[titles]{tocloft}
\cftpagenumbersoff{figure}
\renewcommand{\cftfigpresnum}{\itshape\figurename\enspace}
\renewcommand{\cftfigaftersnum}{.\space}
\setlength{\cftfigindent}{0pt}
\setlength{\cftafterloftitleskip}{0pt}
\settowidth{\cftfignumwidth}{Figure 10.\qquad}
\usepackage[labelfont=bf, font={scriptsize, color=gray}]{caption}
\ifLuaTeX
  \usepackage{selnolig}  % disable illegal ligatures
\fi
\IfFileExists{bookmark.sty}{\usepackage{bookmark}}{\usepackage{hyperref}}
\IfFileExists{xurl.sty}{\usepackage{xurl}}{} % add URL line breaks if available
\urlstyle{same}
\hypersetup{
  pdftitle={The Revised Interoceptive Accuracy Scale (IAS-R) Confirms Links between Interoception and Personality and Psychopathological Traits},
  pdfauthor={Dominique Makowski1, An Shu Te2, \& S.H. Annabel Chen2, 3, 4, 5},
  pdflang={en-EN},
  pdfkeywords={IAS-R; Interoception; Personality; Psychopathology; Validation},
  hidelinks,
  pdfcreator={LaTeX via pandoc}}

\title{\textbf{The Revised Interoceptive Accuracy Scale (IAS-R) Confirms Links between Interoception and Personality and Psychopathological Traits}}
\author{Dominique Makowski\textsuperscript{1}, An Shu Te\textsuperscript{2}, \& S.H. Annabel Chen\textsuperscript{2, 3, 4, 5}}
\date{}


\shorttitle{IAS-R}

\authornote{

The authors made the following contributions. Dominique Makowski: Conceptualization, Data curation, Formal Analysis, Funding acquisition, Investigation, Methodology, Project administration, Resources, Software, Supervision, Validation, Visualization, Writing -- original draft; An Shu Te: Project administration, Resources, Software, Investigation, Writing -- original draft; S.H. Annabel Chen: Project administration, Supervision, Writing -- review \& editing.

Correspondence concerning this article should be addressed to Dominique Makowski, HSS 04-18, 48 Nanyang Avenue, Singapore. E-mail: \href{mailto:dom.makowski@gmail.com}{\nolinkurl{dom.makowski@gmail.com}}

}

\affiliation{\vspace{0.5cm}\textsuperscript{1} School of Psychology, University of Sussex, UK\\\textsuperscript{2} School of Social Sciences, Nanyang Technological University, Singapore\\\textsuperscript{3} LKC Medicine, Nanyang Technological University, Singapore\\\textsuperscript{4} National Institute of Education, Singapore\\\textsuperscript{5} Centre for Research and Development in Learning, Nanyang Technological University, Singapore}

\abstract{%
Something something.
}



\begin{document}
\maketitle

\hypertarget{introduction}{%
\section{Introduction}\label{introduction}}

Interoception - definition - is a trending topic, becoming a central mechanism of embodied cognition, and showing a key contribution to higher order processes such as \ldots.
Unfortunately, it is also notably hard to measure.

Scales are useful to capture metacognitive and subjective aspects and beliefs
While the relationship between interoceptive scales and tasks is a strong point of contention, it is important to continue developing sound scales from a structural (i.e., factorial) standpoint.

One of the most recent scale developped is the IAS, which is interesting because\ldots{} It has \emph{n} items, such as \ldots{}
However, the study analysis did not go in depth in the analysis of the factor structure, focusing instead on \ldots.

The purpose of this work is to reanalyse the factor structure of the scale using complementary statistical approaches, a further extend its validation by comparing a revised version with one of the most popular questionnaire of interoception, the MAIA-2 (\emph{ref}), as well as other relevant indices (such as psychopathological traits).

\hypertarget{study-1---reanalysis-of-murphy-et-al.-2020}{%
\section{Study 1 - Reanalysis of Murphy et al.~(2020)}\label{study-1---reanalysis-of-murphy-et-al.-2020}}

Study 1 is a reanalysis of the data from Murphy et al. (2020) regarding the factor structure of the Interoceptive Accuracy Scale (IAS). The aim is to use a more fine-grained method for estimating the optimal number of latent factors (namely, the \emph{Method Agreement Procedure}, in Lüdecke et al., 2020; Makowski, 2018), and perform a statistical model comparison using Confirmatory Factor Analysis (CFA).

\hypertarget{participants}{%
\subsection{Participants}\label{participants}}

The exploratory factor analysis (EFA) and initial model selection was performed on the data from \href{https://osf.io/3m5nh/?view_only=a68051df4abe4ecb992f22dc8c17f769}{study 1} of Murphy et al. (2020), downloaded from OSF, included 451 participants (Mean age = 25.8, SD = 8.4, range: {[}18, 69{]}; Gender: 69.4\% women, 29.5\% men, 1.11\% non-binary). Data from the \href{https://osf.io/3m5nh/?view_only=a68051df4abe4ecb992f22dc8c17f769}{study 6}, which included 375 participants (Mean age = 35.3, SD = 16.9, range: {[}18, 91{]}; Gender: 70.1\% women, 28.5\% men, 1.33\% non-binary), was used as a test-set for confirmatory analysis.

\hypertarget{results}{%
\subsection{Results}\label{results}}

The \emph{Method Agreement Procedure} suggested 1 latent factor as optimal, supported by 5 (31.25\%) out of 16 methods (Bentler, Acceleration factor, Scree - R2, VSS complexity 1, Velicer's MAP), followed by 4 factors supported by 4 methods (Kaiser criterion, beta score, optimal coordinates, parallel analysis).

We fitted the simple-structure (i.e., each variable loading only unto its maximal latent factor) of these two models using CFA, which underlined the 4-factors model as having a significantly better fit (\(\Delta \chi^2(6) = 232, p < .001; BIC_{CFA-1} = 25141, BIC_{CFA-4} = 24945\)). Using the EFA loading patterns and the CFA modification indices, we then compared the initial 4-factor model to two variants: one with 2 items removed (Blood sugar and Taste), and another with, additionally, the \emph{Interoception} factor split into two (with the pain-related items grouped together in a \emph{Nociception} factor). The latter model (\emph{CFA-5}), was significantly superior to the others (\(\Delta \chi^2(4) = 28.8, p < .001; BIC_{CFA-4mod} = 22563, BIC_{CFA-5} = 22559\)). We then removed the least loaded item of \emph{Expulsion} (cough) to improve the balance (6 items for interoception and 3 per secondary scales), which significantly improved the model fit (\(\Delta \chi^2(17) = 61.4, p < .001; BIC_{CFA-5mod} = 21256\)). Finally, we tested a hierarchical model in which the interoception dimension was, in addition to its own 6 items, loaded by the secondary latent factors (\emph{Skin}, \emph{Expulsion}, \emph{Nociception}, \emph{Elimination}), but this model did not improve the fit (\(BIC_{CFA-5h} = 21452\)).

Finally, we re-fitted the CFA models on a new dataset (sample 2, from study 6 of Murphy et al., 2020), which confirmed that the 5-factor balanced model had the better fit relative to the other models (\(BIC_{CFA-5mod} = 17316; BIC_{CFA-5h} = 17323; BIC_{CFA-5} = 18257; BIC_{EFA-4mod} = 18265\)). However, the final model had poor to barely acceptable absolute indices of fit in both samples (Sample 1: \(RMSEA = .067\) (acceptable \textless{} .08), \(CFI = .878\) (acceptable \textgreater{} .9), \(SRMR = .057\) (acceptable \textless{} .08); Sample 2: \(RMSEA = .068\), \(CFI = .906\), \(SRMR = .063\)).

\hypertarget{summary}{%
\subsection{Summary}\label{summary}}

Exploratory Factor Analysis suggested a 1-factor and 4-factors solutions, but the latter was favoured by CFA. Further comparison suggested that a 5-factors model (obtained by separating \emph{Nociception} from \emph{Interoception} and balancing the number of items per dimension) had a superior fit. The 5 factors (with their items) are \textbf{Interoception} (Heart, Hungry, Breathing, Thirsty, Temperature, Sexual arousal); \textbf{Nociception} (Muscles, Bruise, Pain); \textbf{Expulsion} (Burp, Sneeze, Wind); \textbf{Elimination} (Vomit, Defecate, Urinate); \textbf{Skin} (Itch, Tickle, Affective touch). However, absolute indices of fit for this model were relatively low.

\hypertarget{study-2---online-validation-of-new-structure}{%
\section{Study 2 - Online validation of new structure}\label{study-2---online-validation-of-new-structure}}

The revised scale, made of 18 items (6 for interoception and 3 per secondary dimension), was administered to a new sample in an online study. The response was changed from a 5-point Likert scale (\textbf{ANSHU TO CONFIRM}) to a analog scale.

\hypertarget{participants-1}{%
\subsection{Participants}\label{participants-1}}

485 participants (Mean age = 30.1, SD = 10.1, range: {[}18, 73{]}; Sex: 50.3\% females, 49.7\% males).
\textbf{Update based on para from PHQ4}

Note that as the MAIA and the PI were done on a different session 2, we only have n = \ldots{} for their correlations.

\hypertarget{results-1}{%
\subsection{Results}\label{results-1}}

Despite changing the response format to analog scales, the distribution of answers was similar to that of the original validation samples, with a modal answer at around 75\% of the scale for most of the items (see \textbf{Figure 1A}). Contrary to our expectations, the \emph{Method Agreement Procedure} for EFA suggested 4 latent factor (supported by 5/16 methods), rather than the 5 hypothesized (supported by 0 methods). CFA confirmed that the 4-factor model derived from the EFA had a better fit than this 5-factor model and than a 1-factor model (\(BIC_{CFA-4} = 22758, BIC_{CFA-1} = 22911, BIC_{CFA-1} = 23065\)). We then re-balanced the 4-factor model to keep 3 items per factor and remove items that strongly loaded on more than one factor in the EFA (e.g., Temperature and Sneeze, see \textbf{Figure 1B}). Finally, we compared the resulting model to one with a fifth latent factor (\emph{Interoception}) loaded by the 4 others latent factors. Adding this general score did not significantly change the model's fit (\(\Delta \chi^2(2) = 5.85, p = .054; BIC_{CFA-4mod} = 15140, BIC_{CFA-4h} = 15259\)). Importantly, the resulting model (see \textbf{Figure 1C}) had excellent indices of fit (\(RMSEA = .0347\), \(CFI = .9796\), \(SRMR = .0364\)). Finally, we re-fitted this model on the two samples from study 1, which revealed much improved indices of fit over the initial best model (Sample 1: \(RMSEA = .068\), \(CFI = .9020\), \(SRMR = .0556\); Sample 2: \(RMSEA = .066\), \(CFI = .9315\), \(SRMR = .0504\)).

We ran a Bayesian correlations analysis (with a narrow prior centred around 0)
between the individual facet scores extracted from the final model and the MAIA-2 dimensions (administered online on a different session). All dimensions of the IAS-R correlated with all the MAIA facets with the notable exception of \emph{Not-reacting} and \emph{Not-worrying}. As all correlations are presented in \textbf{Figure 2}, we will focus in the following on a subset of key results.

Correlations with the IPIP6 personality scale highlighted a positive relationship between the \emph{Homeostatis} interoceptive dimension with \emph{Agreeableness} (\(r_{homeostasis} = .14, BF = 17.91\)) and \emph{Conscientiousness} (\(TODO\)); and a negative relationship with \emph{Honesty-Humility} (\(TODO\)) and \emph{Neuroticism} (\(TODO\)). This facet also negatively related to several pathological personality traits measured by the PID-5, such as \emph{Psychoticism} (\(TODO\)), \emph{Negative Affect} (\(TODO\)), and \emph{Detachment} (\(TODO\)). In line with that, we also report negative relationships with schizotypical characteristics, including \emph{Social Anxiety} (\(TODO\)), \emph{Odd Speech}(\(TODO\)), \emph{No Close Friends} (\(TODO\)) and \emph{Constricted Affect} (\(TODO\)); as well with autistic traits, including as \emph{Switching} (\(TODO\)) and \emph{Social Skills} (\(TODO\)).

\hypertarget{summary-1}{%
\subsection{Summary}\label{summary-1}}

\hypertarget{study-3---replication-and-convergent-validity}{%
\section{Study 3 - Replication and Convergent Validity}\label{study-3---replication-and-convergent-validity}}

\hypertarget{general-discussion}{%
\section{General Discussion}\label{general-discussion}}

Whether this scale truly captures interoceptive accuracy is still a matter of debate (as is whether accuracy should even be focus of interoception research, see \textbf{recent review paper that I shared}). In any case, this scale has some advantages over others in that the items are straightforward and do directly relate to bodily processes, without being conflated (at least in their formulation) with emotional or attentional aspects. Despite being at first glance very different from the other personality measures, we found consistent and strong relationships.

The fact that the \emph{Homeostatis} dimension was the most significantly correlated with other subjective measures could be that it captures the most overt and key features of bodily signals (that relates to primal needs). As such, it might be the subscale with the most meaningful variability and accuracy. However, the relationships observed with this subscale were also consistently present for the other subscales (though typically with a lesser magnitude).

\newpage

\hypertarget{revised-interoceptive-accuracy-scale-ias-r}{%
\section{Revised Interoceptive Accuracy Scale (IAS-R)}\label{revised-interoceptive-accuracy-scale-ias-r}}

\textbf{Scoring}. Each scale can be answered on an analogue scale (Disagree - Agree). Items can be averaged per dimension, and dimensions can be averaged to get a general Interoception score.

\textbf{Instructions}. ``Below are several statements regarding how accurately you can perceive specific bodily sensations. Please rate on the scale on how well you believe you can perceive each specific signal. For example, if you often feel you need to urinate and then realise you do not need to when you go to the toilet you would rate your accuracy perceiving this bodily signal as low. Please only rate how well you can perceive these signals without using external cues. For example, if you can only perceive how fast your heart is beating when you measure it by taking your pulse this would not count as accurate internal perception''.

\begin{table}[!h]

\begin{center}
\begin{threeparttable}

\caption{\label{tab:unnamed-chunk-2}}

\scriptsize{

\begin{tabular}{ll}
\toprule
Facet & Item\\
\midrule
\textbf{Anxiety} & Breathing\\
 & Muscles\\
 & I can always accurately perceive when my heart is beating fast\\ \midrule
\textbf{Homeostatis} & Itch\\
 & Tickle\\
 & Bruise\\
\bottomrule
\end{tabular}

}

\end{threeparttable}
\end{center}

\end{table}

\newpage

\hypertarget{data-availability}{%
\section{Data Availability}\label{data-availability}}

The dataset analysed during the current study are available in the GitHub repository \url{https://github.com/DominiqueMakowski/InteroceptiveAccuracyScale}.

\hypertarget{funding}{%
\section{Funding}\label{funding}}

This work was supported by the Presidential Postdoctoral Fellowship Grant (NTU-PPF-2020-10014) from Nanyang Technological University (awarded to DM).

\hypertarget{acknowledgments}{%
\section{Acknowledgments}\label{acknowledgments}}

We warmly thank the original authors of (Murphy et al., 2020) for making their data and material open-access, which enabled the present follow-up study.

\newpage

\hypertarget{references}{%
\section{References}\label{references}}

\hypertarget{refs}{}
\begin{CSLReferences}{1}{0}
\leavevmode\vadjust pre{\hypertarget{ref-ludecke2020extracting}{}}%
Lüdecke, D., Ben-Shachar, M. S., Patil, I., \& Makowski, D. (2020). Extracting, computing and exploring the parameters of statistical models using r. \emph{Journal of Open Source Software}, \emph{5}(53), 2445.

\leavevmode\vadjust pre{\hypertarget{ref-makowski2018psycho}{}}%
Makowski, D. (2018). The psycho package: An efficient and publishing-oriented workflow for psychological science. \emph{Journal of Open Source Software}, \emph{3}(22), 470.

\leavevmode\vadjust pre{\hypertarget{ref-murphy2020testing}{}}%
Murphy, J., Brewer, R., Plans, D., Khalsa, S. S., Catmur, C., \& Bird, G. (2020). Testing the independence of self-reported interoceptive accuracy and attention. \emph{Quarterly Journal of Experimental Psychology}, \emph{73}(1), 115--133.

\end{CSLReferences}


\clearpage
\renewcommand{\listfigurename}{Figure captions}


\end{document}
